%!TEX TS-program = xelatex
\documentclass[12pt, a4paper, oneside]{article}

% пакеты для математики
\usepackage{amsmath,amsfonts,amssymb,amsthm,mathtools}  
\mathtoolsset{showonlyrefs=true}  % Показывать номера только у тех формул, на которые есть \eqref{} в тексте.

\usepackage[british,russian]{babel} % выбор языка для документа
\usepackage[utf8]{inputenc}          % utf8 кодировка

% Основные шрифты 
\usepackage{fontspec}         
\setmainfont{Linux Libertine O}  % задаёт основной шрифт документа

% Математические шрифты 
\usepackage{unicode-math}     
\setmathfont[math-style=upright]{Neo Euler} 

%%%%%%%%%% Работа с картинками и таблицами %%%%%%%%%%
\usepackage{graphicx} % Для вставки рисунков                
\usepackage{graphics}
\graphicspath{{images/}{pictures/}}   % папки с картинками

\usepackage[figurename=Картинка]{caption}

\usepackage{wrapfig}    % обтекание рисунков и таблиц текстом

\usepackage{booktabs}   % таблицы как в годных книгах
\usepackage{tabularx}   % новые типы колонок
\usepackage{tabulary}   % и ещё новые типы колонок
\usepackage{float}      % возможность позиционировать объекты в нужном месте
\renewcommand{\arraystretch}{1.2}  % больше расстояние между строками


%%%%%%%%%% Графики и рисование %%%%%%%%%%
\usepackage{tikz, pgfplots}  % языки для графики
%\pgfplotsset{compat=1.16}

\usepackage{todonotes} % для вставки в документ заметок о том, что осталось сделать
% \todo{Здесь надо коэффициенты исправить}
% \missingfigure{Здесь будет Последний день Помпеи}
% \listoftodos --- печатает все поставленные \todo'шки


%%%%%%%%%% Внешний вид страницы %%%%%%%%%%

\usepackage[paper=a4paper, top=20mm, bottom=15mm,left=20mm,right=15mm]{geometry}
\usepackage{indentfirst}    % установка отступа в первом абзаце главы

\usepackage{setspace}
\setstretch{1.15}  % межстрочный интервал
\setlength{\parskip}{4mm}   % Расстояние между абзацами
% Разные длины в LaTeX: https://en.wikibooks.org/wiki/LaTeX/Lengths

% свешиваем пунктуацию
% теперь знаки пунктуации могут вылезать за правую границу текста, при этом текст выглядит ровнее
\usepackage{microtype}

% \flushbottom                            % Эта команда заставляет LaTeX чуть растягивать строки, чтобы получить идеально прямоугольную страницу
\righthyphenmin=2                       % Разрешение переноса двух и более символов
\widowpenalty=300                     % Небольшое наказание за вдовствующую строку (одна строка абзаца на этой странице, остальное --- на следующей)
\clubpenalty=3000                     % Приличное наказание за сиротствующую строку (омерзительно висящая одинокая строка в начале страницы)
\tolerance=10000     % Ещё какое-то наказание.

% мои цвета https://www.artlebedev.ru/colors/
\definecolor{titleblue}{rgb}{0.2,0.4,0.6} 
\definecolor{blue}{rgb}{0.2,0.4,0.6} 
\definecolor{red}{rgb}{1,0,0.2} 
\definecolor{green}{rgb}{0,0.6,0} 
\definecolor{purp}{rgb}{0.4,0,0.8} 

% цвета из geogebra 
\definecolor{litebrown}{rgb}{0.6,0.2,0}
\definecolor{darkbrown}{rgb}{0.75,0.75,0.75}

% Гиперссылки
\usepackage{xcolor}   % разные цвета

\usepackage{hyperref}
\hypersetup{
	unicode=true,           % позволяет использовать юникодные символы
	colorlinks=true,       	% true - цветные ссылки
	urlcolor=blue,          % цвет ссылки на url
	linkcolor=black,          % внутренние ссылки
	citecolor=green,        % на библиографию
	breaklinks              % если ссылка не умещается в одну строку, разбивать её на две части?
}

% меняю оформление секций 
\usepackage{titlesec}
\usepackage{sectsty}

% меняю цвет на синий
\sectionfont{\color{titleblue}}
\subsectionfont{\color{titleblue}}

% выбрасываю нумерацию страниц и колонтитулы 
\pagestyle{empty}

% синие круглые бульпоинты в списках itemize 
\usepackage{enumitem}

\definecolor{itemizeblue}{rgb}{0, 0.45, 0.70}

\newcommand*{\MyPoint}{\tikz \draw [baseline, fill=itemizeblue, draw=blue] circle (2.5pt);}
\renewcommand{\labelitemi}{\MyPoint}

\AddEnumerateCounter{\asbuk}{\@asbuk}{\cyrm}
\renewcommand{\theenumi}{\asbuk{enumi}}

% расстояние в списках
\setlist[itemize]{parsep=0.4em,itemsep=0em,topsep=0ex}
\setlist[enumerate]{parsep=0.4em,itemsep=0em,topsep=0ex}

% эпиграфы
\usepackage{epigraph}
\setlength\epigraphwidth{.6\textwidth}
\setlength\epigraphrule{0pt}

%%%%%%%%%% Свои команды %%%%%%%%%%

% Математические операторы первой необходимости:
\DeclareMathOperator{\sgn}{sign}
\DeclareMathOperator*{\argmin}{arg\,min}
\DeclareMathOperator*{\argmax}{arg\,max}
\DeclareMathOperator{\Cov}{Cov}
\DeclareMathOperator{\Var}{Var}
\DeclareMathOperator{\Corr}{Corr}
\DeclareMathOperator{\E}{\mathop{E}}
\DeclareMathOperator{\Med}{Med}
\DeclareMathOperator{\Mod}{Mod}
\DeclareMathOperator*{\plim}{plim}

% команды пореже
\newcommand{\const}{\mathrm{const}}  % const прямым начертанием
\newcommand{\iid}{\sim i.\,i.\,d.}  % ну вы поняли...
\newcommand{\fr}[2]{\ensuremath{^{#1}/_{#2}}}   % особая дробь
\newcommand{\ind}[1]{\mathbbm{1}_{\{#1\}}} % Индикатор события
\newcommand{\dx}[1]{\,\mathrm{d}#1} % для интеграла: маленький отступ и прямая d

% одеваем шапки на частые штуки
\def \hb{\hat{\beta}}
\def \hs{\hat{s}}
\def \hy{\hat{y}}
\def \hY{\hat{Y}}
\def \he{\hat{\varepsilon}}
\def \hVar{\widehat{\Var}}
\def \hCorr{\widehat{\Corr}}
\def \hCov{\widehat{\Cov}}

% Греческие буквы
\def \a{\alpha}
\def \b{\beta}
\def \t{\tau}
\def \dt{\delta}
\def \e{\varepsilon}
\def \ga{\gamma}
\def \kp{\varkappa}
\def \la{\lambda}
\def \sg{\sigma}
\def \tt{\theta}
\def \Dt{\Delta}
\def \La{\Lambda}
\def \Sg{\Sigma}
\def \Tt{\Theta}
\def \Om{\Omega}
\def \om{\omega}

% Готика
\def \mA{\mathcal{A}}
\def \mB{\mathcal{B}}
\def \mC{\mathcal{C}}
\def \mE{\mathcal{E}}
\def \mF{\mathcal{F}}
\def \mH{\mathcal{H}}
\def \mL{\mathcal{L}}
\def \mN{\mathcal{N}}
\def \mU{\mathcal{U}}
\def \mV{\mathcal{V}}
\def \mW{\mathcal{W}}

% Жирные буквы
\def \mbb{\mathbb}
\def \RR{\mbb R}
\def \NN{\mbb N}
\def \ZZ{\mbb Z}
\def \PP{\mbb{P}}
\def \QQ{\mbb Q}

%%%%%%%%%% Теоремы %%%%%%%%%%
\theoremstyle{plain} % Это стиль по умолчанию.  Есть другие стили.
\newtheorem{theorem}{Теорема}[section]
\newtheorem{result}{Следствие}[theorem]
% счётчик подчиняется теоремному, нумерация идёт по главам согласованно между собой

% убирает курсив и что-то еще наверное делает ;)
\theoremstyle{definition}         
\newtheorem*{definition}{Определение}  % нумерация не идёт вообще


%%%%%%%%%% Задачки и решения %%%%%%%%%%
\usepackage{etoolbox}    % логические операторы для своих макросов
\usepackage{environ}
\newtoggle{lecture}

\newcounter{probNum}[section]  % счётчик для упражнений 
\NewEnviron{problem}[1]{%
	\refstepcounter{probNum}% увеличели номер на 1 
	{\noindent \textbf{\large \color{titleblue} Упражнение~\theprobNum~#1}  \\ \\ \BODY}
	{}%
}

% Окружение, чтобы можно было убирать решения из pdf
\NewEnviron{sol}{%
	\iftoggle{lecture}
	{\noindent \textbf{\large Решение:} \\ \\ \BODY}
	{}%
}

% выделение по тексту важных вещей
\newcommand{\indef}[1]{\textbf{ \color{green} #1}} 

% разные дополнения для картинок
\usetikzlibrary{arrows.meta}
\usepackage{varwidth}

% рисование крестов
% https://tex.stackexchange.com/questions/123760/draw-crosses-in-tikz
\tikzset{
	cross/.pic = {
		\draw[line width=2.pt, rotate = 45] (-#1,0) -- (#1,0);
		\draw[line width=2.pt, rotate = 45] (0,-#1) -- (0, #1);
	}
}

\usepackage[normalem]{ulem}  % для зачекивания текста
\usepackage{tikz, pgfplots}  % язык для рисования графики из latex'a

\usepackage{todonotes} % для вставки в документ заметок о том, что осталось сделать
% \todo{Здесь надо коэффициенты исправить}
% \missingfigure{Здесь будет Последний день Помпеи}
% \listoftodos --- печатает все поставленные \todo'шки

\title{%
	Тятя! Тятя! Наши сети притащили мертвеца! \\
	\large Листочек с задачками №1: всего лишь функция
}
\date{РАНХ \\ осень 2020}
\author{\url{https://github.com/FUlyankin/neural_nets_prob} }

\begin{document}
	
	\maketitle
	
\epigraph{Ты всего лишь машина, только имитация жизни. Робот сочинит симфонию? Робот превратит кусок холста в шедевр искусства?}{Из фильма <<Я, робот>> (2004)}

%%%-------------------------------------------
\begin{problem}{(от регрессии к нейросетке)}
	Однажды вечером, по пути с работы\footnote{она работает рисёрчером.} Маша зашла в свою любимую кофейню на Тверской. Там, на стене, она обнаружила очень интересную картину:
	
	\begin{center}
		\begin{tikzpicture}[line cap=round,line join=round,x=1.0cm,y=1.0cm]
		
		\draw [line width=1.pt] (0,1.5) circle (0.5cm) node {$1$};
		\draw [line width=1.pt] (2,1.5) circle (0.5cm) node {$x$};
		\draw [line width=1.pt] (1,-1) circle (0.5cm) node {$y$};
		
		\draw [->, line width=1.pt] (0,1) -- (0.9,-0.4) node[pos=0.3,right] {\small $w_0$};
		\draw [->, line width=1.pt] (2,1) -- (1.1,-0.4) node[pos=0.3,left] {\small $w_1$};
		\end{tikzpicture}
	\end{center}
	
	Хозяин кофейни, Добродум, объяснил Маше, что это Покрас-Лампас так нарисовал линейную регрессию,\footnote{экслюзивный заказ был} и её легко можно переписать в виде формулы: $y_i = w_0 + w_1 \cdot x_i.$ Пока Добродум готовил кофе, Маша накидала у себя на бумажке новую картинку: 
	
	\begin{center}
		\begin{tikzpicture}[line cap=round,line join=round,x=1.0cm,y=1.0cm]
		
		\draw [line width=1.pt] (0,2.5) circle (0.5cm) node {$1$};
		\draw [line width=1.pt] (2,2.5) circle (0.5cm) node {$x$};
		
		\draw [line width=1.pt] (0,-0.5) circle (0.5cm) node {$h_1$};
		\draw [line width=1.pt] (2,-0.5) circle (0.5cm) node {$h_2$};
		
		\draw [line width=1.pt] (1,-3) circle (0.5cm) node {$y$};
		
		\draw [->, line width=1.pt] (0,2) -- (-0.1,0.1) node[pos=0.3,left] {\small $w_{11}$};
		\draw [->, line width=1.pt] (2,2) -- (2.1,0.1) node[pos=0.3,right] {\small $w_{22}$};
		
		\draw [->, line width=1.pt] (0,2) -- (1.9,0.1) node[pos=0.5,left] {\small $w_{12}$};
		\draw [->, line width=1.pt] (2,2) -- (0.1,0.1) node[pos=0.5,right] {\small $w_{21}$};
		
		\draw [->, line width=1.pt] (0,-1) -- (0.8,-2.4) node[pos=0.3,right] {\small $w_{1}$};
		\draw [->, line width=1.pt] (2,-1) -- (1.2,-2.4) node[pos=0.3,left] {\small $w_{2}$};
		\end{tikzpicture}
	\end{center}
	
	\begin{enumerate} 
		\item Как такая функция будет выглядеть в виде формулы? 
		\item Правда ли, что $y$ будет нелинейно зависеть от $x$?
		\item Если нет, как это исправить и сделать зависимость нелинейной? 
		% \item Запишите для трёх наблюдений в матричной форме сначала регрессию, а потом новую функцию Маши. 
	\end{enumerate} 
\end{problem}


%%%-------------------------------------------
\begin{problem}{(из картинки в формулу)}
	Добродум хочет понять насколько сильно будет заполнена кофейня в следующие выходные. Для этого он обучил нейросетку. На вход она принимает три фактора: температуру за окном, $x_1$, факт наличия на Тверской митинга, $x_2$ и пол баристы на смене, $x_3$.  В качестве функции активации Добродум использует $ReLU.$ 
	
	\begin{center}
		\begin{tikzpicture}[line cap=round,line join=round,x=1.0cm,y=1.0cm]
		
		\draw [line width=1.pt] (-3,1.5) circle (0.5cm) node {$x_1$};
		\draw [line width=1.pt] (-3,-0.5) circle (0.5cm) node {$x_2$};
		\draw [line width=1.pt] (-3,-2.5) circle (0.5cm) node {$x_3$};	
		
		\draw [line width=1.pt] (-1,0)--(1,0)--(1,1)--(-1,1)--cycle;
		\draw (0,0.5) node {$f(h)$};
		\draw [line width=1.pt] (-1,-2)--(1,-2)--(1,-1)--(-1,-1)--cycle;
		\draw (0,-1.5) node {$f(h)$};
		
		\draw [line width=1.pt] (3,1.5) circle (0.5cm) node {$1$};	
		\draw [line width=1.pt] (2,-1)--(4,-1)--(4,0)--(2,0)--cycle;
		\draw (3,-0.5) node {$f(h)$};
		
		\draw [line width=1.pt] (-2.5,1.5) -- (-1,0.5) node[pos=0.5,above] {\small $5$};
		\draw [line width=1.pt] (-2.5,1.5) -- (-1,-1.5) node[pos=0.3,left] {\small $-2$};
		\draw [line width=1.pt] (-2.5,-0.5) -- (-1,0.5) node[pos=0.5,left] {\small $-60$};
		\draw [line width=1.pt] (-2.5,-0.5) -- (-1,-1.5)  node[pos=0.5,left] {\small  $-80$};
		\draw [line width=1.pt] (-2.5,-2.5) -- (-1,0.5) node[near start,left] {\small $0.5$};
		\draw [line width=1.pt] (-2.5,-2.5) -- (-1,-1.5)  node[near start,below] {\small  $2$};
		
		\draw [line width=1.pt] (1,0.5) -- (2,-0.5) node[pos=0.25,right] {\small $1$};
		\draw [line width=1.pt] (1,-1.5) -- (2,-0.5) node[pos=0.2,right] {\small $0.2$};
		\draw [line width=1.pt] (3,1) -- (3,0) node[pos=0.5,right] {\small  $4$};
		
		\draw [->] (4,-0.5) -- (5,-0.5) node[right] {$\hat y$};
		\end{tikzpicture}
	\end{center}
	
	\begin{enumerate}
		\item В эти выходные за барной\footnote{барной... конечно, кофейня у него...} стойкой стоит Агнесса. Митинга не предвидится, температура будет в районе $20$ градусов. Сколько человек придёт в кофейню к Добродуму? 
		
		\item На самом деле каждая нейросетка --- это просто-напросто какая-то нелинейная сложная функция. Запишите нейросеть Добродума в виде функции.
	\end{enumerate}
\end{problem}


%%%-------------------------------------------
\begin{problem}{(из формулы в картинку)}
	Маша написала на бумажке функцию: 
	
	\begin{equation*}
	y = \max(0, 4 \cdot \max(0, 3 \cdot x_1 + 4 \cdot x_2 + 1) + 2 \cdot \max(0, 3 \cdot x_1 + 2 \cdot x_2 + 7) + 6)
	\end{equation*} 
	
	Теперь она хочет, чтобы кто-нибудь из её адептов нарисовал её в виде нейросетки. Нарисуй.
\end{problem}

\begin{problem}{(армия регрессий)}
	Парни очень любят Машу,\footnote{когда у тебя есть лёрнинг, они так и лезут} а Маша с недавних пор любит собирать перcептроны и думать по вечерам об их весах и функциях активации. Сегодня она решила разобрать свои залежи из перcептронов и как следует упорядочить их. 
	
	\begin{enumerate}
		\item В ящике стола Маша нашла перcептрон с картинки \ref{perp1} Маша хочет подобрать веса так, чтобы он реализовывал логическое отрицание, то есть превращал $x_1 = 0$ в $y=1$, а $x_1 = 1$ в $y=0$ (так работает логичская функция: отрицание). 
		
		\begin{minipage}{0.49\linewidth}
			\begin{figure}[H]
				\caption{} \label{perp1}
				\begin{center}
					\begin{tikzpicture}[line cap=round,line join=round,x=1.0cm,y=1.0cm]
					\clip(-4,0) rectangle (3,4.5); % чтобы выравнять картинки по размерам
					\draw [line width=1.pt] (-3,4) circle (0.5cm) node {$1$};
					\draw [line width=1.pt] (-3,2) circle (0.5cm) node {$x_1$};
					
					\draw [line width=1.pt] (-1,3.5)--(-1,2.5)--(1.5,2.5)-- (1.5,3.5) -- cycle;
					
					\draw [->, line width=1.pt] (-2.5,4) -- (-1.1,3.1) node[pos=0.3,above] {\small $w_1$};
					\draw [->, line width=1.pt] (-2.5,2) -- (-1.1,2.9) node[pos=0.3,above] {\small $w_2$};
					\draw [->, line width=1.pt] (1.5,3) -- (2.5,3) node[pos=1,right] {$\hat y$};
					\draw (0.3,3) node {$\max(0,h)$};
					\end{tikzpicture}
				\end{center}
			\end{figure}
		\end{minipage}
		\hfill
		\begin{minipage}{0.49\linewidth}
			\begin{figure}[H]
				\caption{} \label{perp2}
				\begin{center}
					\begin{tikzpicture}[line cap=round,line join=round,x=1.0cm,y=1.0cm]
					\clip(-4,0) rectangle (3,4.5);
					\draw [line width=1.pt] (-3,4) circle (0.5cm) node {$x_1$};
					\draw [line width=1.pt] (-3,2.5) circle (0.5cm) node {$x_2$};
					\draw [line width=1.pt] (-3,1) circle (0.5cm) node {$x_3$};
					
					\draw [line width=1.pt] (-1,3)--(-1,2)--(1.5,2)--(1.5,3)--cycle;
					\draw (0.3,2.5) node {$\max(0,h)$};
					
					\draw [->, line width=1.pt] (-2.5,4)--(-1.1,2.7) node[pos=0.5,above] {\small $w_1$};
					\draw [->, line width=1.pt] (-2.5,2.5)--(-1.1,2.5) node[pos=0.3,above] {\small $w_2$};
					\draw [->, line width=1.pt] (-2.5,1)--(-1.1,2.3) node[pos=0.5,below] {\small $w_3$};
					\draw [->, line width=1.pt] (1.5,2.5)--(2.5,2.5) node[pos=1,right] {$\hat y$};
					\end{tikzpicture}
				\end{center}
			\end{figure}
		\end{minipage}
		
		\item В тумбочке, среди носков, Маша нашла перcептрон, с картинки \ref{perp2}, Маша хочет подобрать такие веса $w_i$, чтобы персептрон превращал $x$ из таблички в соответствующие $y$:
		
		\begin{center}
			\begin{tabular}{c|c|c|c}
				$x_1$ & $x_2$ & $x_3$ & $y$ \\
				\hline 
				$1$ & $1$ & $2$ & $0.5$\\
				\hline 
				$1$ & $-1$ & $1$ & $0$ \\
			\end{tabular}
		\end{center}
		
		\item Оказывается, что в ванной всё это время валялась куча персептронов с картинки \ref{perp3} с неизвестной функцией активации (надо самому выбирать). 
		
		\begin{minipage}{0.49\linewidth}
			\begin{figure}[H]
				\caption{} \label{perp3}
				\begin{center}
					\begin{tikzpicture}[line cap=round,line join=round,x=1.0cm,y=1.0cm]
					\clip(-4,0) rectangle (3,4.5);
					\draw [line width=1.pt] (-3,4) circle (0.5cm) node {$1$};
					\draw [line width=1.pt] (-3,2.5) circle (0.5cm) node {$x_1$};
					\draw [line width=1.pt] (-3,1) circle (0.5cm) node {$x_2$};
					
					\draw [line width=1.pt] (-1,3)--(-1,2)--(1.5,2)--(1.5,3)--cycle;
					\draw (0.3,2.5) node {$f(h)$};
					
					\draw [->, line width=1.pt] (-2.5,4)--(-1.1,2.7) node[pos=0.5,above] {\small $w_1$};
					\draw [->, line width=1.pt] (-2.5,2.5)--(-1.1,2.5) node[pos=0.3,above] {\small $w_2$};
					\draw [->, line width=1.pt] (-2.5,1)--(-1.1,2.3) node[pos=0.5,below] {\small $w_3$};
					\draw [->, line width=1.pt] (1.5,2.5)--(2.5,2.5) node[pos=1,right] {$\hat y$};
					\end{tikzpicture}
				\end{center}
			\end{figure}
		\end{minipage}
		\hfill
		\begin{minipage}{0.49\linewidth}
			\begin{figure}[H]
				\caption{} \label{perp4}
				\begin{center}
					\begin{tikzpicture}[line cap=round,line join=round,x=1.0cm,y=1.0cm]
					\clip(-2,-5) rectangle (4,-0.5);
					\draw [->,line width=2.pt] (1,-5) -- (1,-1);
					\draw [->,line width=2.pt] (-2,-3) -- (4,-3);
					\draw (2,-1.25) node[anchor=north west] {$1$};
					\draw (-0.3,-2.4) node[anchor=north west] {$0$};
					\draw (1.9,-4) node[anchor=north west] {$0$};
					\draw (3.2,-2.5) node[anchor=north west] {$0$};	
					\draw [line width=2.pt,dash pattern=on 3pt off 3pt] (0,-5)-- (4,-1);
					\draw [line width=2.pt,dash pattern=on 3pt off 3pt] (0,-1)-- (4,-5);
					\end{tikzpicture}
				\end{center}
			\end{figure}
		\end{minipage}
		
		Маша провела на плоскости две прямые: $x_1 + x_2 = 1$ и $x_1 - x_2 = 1$. Она хочет собрать из персептронов нейросетку, которая будет классифицировать объекты с плоскости так, как показано на картинке \ref{perp4}.
		
		%%%% Дублирует предыдущий пункт :( 
		% \item В коробке на кухне завалялось три персептрона, у каждого два входа с константой и пороговая функция активации:
		
		% \begin{equation*} 
		% f(h) = \begin{cases} 1, h \ge 0 \\ 0, h < 0.\end{cases}
		% \end{equation*} 
		
		% Маша хочет реализовать с их помощью функцию
		
		% $$
		% y = \begin{cases}
		% 1, \text{ если } x_2 \geq |x_1 - 3| + 2; \\
		% 0, \text{ иначе}
		% \end{cases}.
		% $$
	\end{enumerate}
\end{problem}

%%%-------------------------------------------
\begin{problem}{(логические функции)}
	Маша вчера поссорилась с Пашей. Он сказал, что у неё нет логики. Чтобы доказать Паше обратное, Маша нашла теорему, которая говорит о том, что с помощью нейросетки можно аппроксимировать почти любую функцию, и теперь собирается заняться апроксимацией логических функций. Для начала она взяла самые простые, заданные следующими таблицами истинности:
	
	\begin{center}
		\begin{minipage}{0.3\linewidth} 
			\begin{tabular}{c|c|c}
				$x_1$ & $x_2$ & $x_1 \cap x_2$ \\
				\hline 
				$1$ & $1$ & $1$ \\
				\hline 
				$1$ & $0$ & $0$ \\
				\hline 
				$0$ & $1$ & $0$ \\
				\hline 
				$0$ & $0$ & $0$ \\
			\end{tabular}
		\end{minipage}
		\hfill
		\begin{minipage}{0.3\linewidth}
			\begin{tabular}{c|c|c}
				$x_1$ & $x_2$ & $x_1 \cup x_2$ \\
				\hline 
				$1$ & $1$ & $1$ \\
				\hline 
				$1$ & $0$ & $1$ \\
				\hline 
				$0$ & $1$ & $1$ \\
				\hline 
				$0$ & $0$ & $0$ \\
			\end{tabular}
		\end{minipage}
		\hfill
		\begin{minipage}{0.3\linewidth}
			\begin{tabular}{c|c|c}
				$x_1$ & $x_2$ & $x_1 \mbox{ } XoR \mbox{ } x_2$ \\
				\hline 
				$1$ & $1$ & $0$ \\
				\hline 
				$1$ & $0$ & $1$ \\
				\hline 
				$0$ & $1$ & $1$ \\
				\hline 
				$0$ & $0$ & $0$ \\
			\end{tabular}
		\end{minipage}
	\end{center}
	
	Первые два столбика идут на вход, третий получается на выходе. Первая операция --- логическое "и", вторая --- "или".   Операция из третьей таблицы называется "исключающим или" (XoR). Если внимательно приглядеться, то можно заметить, что $XoR$ --- это то же самое что и $[x_1 \ne x_2]$\footnote{Тут квадратные скобки обозначают индикатор. Он выдаёт $1$, если внутри него стоит правда и $0$, если ложь.}. 
\end{problem}


%%%-------------------------------------------
\begin{problem}{(ещё немного про XoR)} 
	Маша заметила, что на $XoR$ ушло очень много персептронов. Она поняла, что первые два персептрона пытаются сварить для третьего нелинейные признаки, которых нейросетке не хватает. Она решила самостоятельно добавить персептрону вход $x_3 = x_1 \cdot x_2$ и реализовать XoR одним персептроном. Можно ли это сделать? 
\end{problem}


%%%-------------------------------------------
\begin{problem}{(универсальный классификатор)}
	Маша задумалась о том, можно ли с помощью нейронной сетки с одним скрытым слоем и ступенчатой функцией активации решить абсолютно любую задачу классификации на два класса со сколь угодно большой точностью. Ей кажется, что да. Как это можно сделать? 
\end{problem} 


%%%-------------------------------------------
\begin{problem}{(минималочка)}
	Шестилетняя сестрёнка ворвалась в квартиру Маши и разрисовала ей все обои: 
	
	\begin{minipage}{0.25\linewidth} 
		\begin{center}
			\begin{tikzpicture}[scale = 0.7, line cap=round,line join=round,x=1.0cm,y=1.0cm]
			\draw [fill=blue] (-0.44,5.84) circle (2.5pt);
			\draw [fill=blue] (-1.52,5.9) circle (2.5pt);
			\draw [fill=blue] (-3.,6.) circle (2.5pt);
			\draw [fill=blue] (-4.02,5.7) circle (2.5pt);
			\draw [fill=blue] (-3.36,4.28) circle (2.5pt);
			\draw [fill=blue] (-2.4,2.92) circle (2.5pt);
			\draw [fill=blue] (-1.18,3.38) circle (2.5pt);
			\draw [fill=blue] (-0.56,4.16) circle (2.5pt);
			\draw [fill=blue] (0.,5.) circle (2.5pt);
			\draw [fill=blue] (-3.52,3.32) circle (2.5pt);
			\draw [fill=blue] (-0.44,2.5) circle (2.5pt);
			\draw [fill=blue] (-1.44,2.78) circle (2.5pt);
			\draw [fill=blue] (0.06,6.38) circle (2.5pt);
			\draw [fill=blue] (-2.96,6.78) circle (2.5pt);
			\draw [fill=blue] (-4.02,4.56) circle (2.5pt);
			\path (-2.06,4.62) pic[red] {cross=2.5pt};
			\path (-2.,5.) pic[red] {cross=2.5pt};	
			\path (-1.72,4.4) pic[red] {cross=2.5pt};
			\path (-2.46,4.42) pic[red] {cross=2.5pt};
			\path (-2.6,4.9) pic[red] {cross=2.5pt};
			\path (-2.,4.) pic[red] {cross=2.5pt};
			\path (-1.42,4.9) pic[red] {cross=2.5pt};
			\end{tikzpicture}
		\end{center} 
	\end{minipage} 
	\hfill
	\begin{minipage}{0.25\linewidth} 
		\begin{center}
			\begin{tikzpicture}[scale = 0.7, line cap=round,line join=round,x=1.0cm,y=1.0cm]
			\draw [fill=blue] (1.84,5.84) circle (2.5pt);
			\draw [fill=blue] (1.84,4.86) circle (2.5pt);
			\draw [fill=blue] (1.84,3.62) circle (2.5pt);
			\draw [fill=blue] (1.84,2.62) circle (2.5pt);
			\draw [fill=blue] (2.68,2.56) circle (2.5pt);
			\draw [fill=blue] (2.68,3.38) circle (2.5pt);
			\draw [fill=blue] (2.54,4.5) circle (2.5pt);
			\draw [fill=blue] (2.64,5.72) circle (2.5pt);
			
			\draw [fill=blue] (5.04,5.6) circle (2.5pt);
			\draw [fill=blue] (5.,4.56) circle (2.5pt);
			\draw [fill=blue] (4.92,3.36) circle (2.5pt);
			\draw [fill=blue] (4.92,2.48) circle (2.5pt);
			\draw [fill=blue] (5.98,2.32) circle (2.5pt);
			\draw [fill=blue] (5.98,3.16) circle (2.5pt);
			\draw [fill=blue] (6.,4.) circle (2.5pt);
			\draw [fill=blue] (5.94,4.76) circle (2.5pt);
			\draw [fill=blue] (5.92,5.34) circle (2.5pt);
			\draw [fill=blue] (5.5,4) circle (2.5pt);
			
			\path (3.42,5.12) pic[red] {cross=2.5pt};	
			\path (3.44,5.94) pic[red] {cross=2.5pt};	
			\path (4.28,5.8) pic[red] {cross=2.5pt};	
			\path (4.28,5.3) pic[red] {cross=2.5pt};	
			\path (3.54,3.98) pic[red] {cross=2.5pt};	
			\path (4.,4.) pic[red] {cross=2.5pt};	
			\path (3.5,2.98) pic[red] {cross=2.5pt};	
			\path (4.18,2.94) pic[red] {cross=2.5pt};	
			\path (3.4,2.32) pic[red] {cross=2.5pt};	
			\path (4.18,2.36) pic[red] {cross=2.5pt};	
			\path (3.98,4.56) pic[red] {cross=2.5pt};	
			
			\path (6.42,5.12) pic[red] {cross=2.5pt};	
			\path (6.44,5.94) pic[red] {cross=2.5pt};	
			\path (7.28,5.8) pic[red] {cross=2.5pt};	
			\path (7.28,5.3) pic[red] {cross=2.5pt};	
			\path (6.54,3.98) pic[red] {cross=2.5pt};	
			\path (7.,4.) pic[red] {cross=2.5pt};	
			\path (6.5,2.98) pic[red] {cross=2.5pt};	
			\path (7.18,2.94) pic[red] {cross=2.5pt};	
			\path (6.4,2.32) pic[red] {cross=2.5pt};	
			\path (7.18,2.36) pic[red] {cross=2.5pt};	
			\path (6.98,4.56) pic[red] {cross=2.5pt};
			\end{tikzpicture}
		\end{center} 
	\end{minipage} 
	\hfill
	\begin{minipage}{0.15\linewidth} 
		\begin{center}
			\begin{tikzpicture}[scale = 0.7, line cap=round,line join=round,x=1.0cm,y=1.0cm]
			\draw [fill=blue] (0,0) circle (2.5pt);
			\draw [fill=blue] (0,0.5) circle (2.5pt);
			\draw [fill=blue] (0,1) circle (2.5pt);
			\draw [fill=blue] (0,1.5) circle (2.5pt);
			\path (0,-1) pic[red] {cross=2.5pt};
			\path (0,-1.5) pic[red] {cross=2.5pt};
			\path (0,-0.5) pic[red] {cross=2.5pt};
			\path (0,-2) pic[red] {cross=2.5pt};
			
			\draw [fill=blue] (0.5,-0.5) circle (2.5pt);
			\draw [fill=blue] (0.5,-1) circle (2.5pt);
			\draw [fill=blue] (0.5,-1.5) circle (2.5pt);
			\draw [fill=blue] (0.5,-2) circle (2.5pt);
			\path (0.5,0) pic[red] {cross=2.5pt};
			\path (0.5,0.5) pic[red] {cross=2.5pt};
			\path (0.5,1) pic[red] {cross=2.5pt};
			\path (0.5,1.5) pic[red] {cross=2.5pt};
			
			\draw [fill=blue] (1,0) circle (2.5pt);
			\draw [fill=blue] (1,0.5) circle (2.5pt);
			\draw [fill=blue] (1,1) circle (2.5pt);
			\draw [fill=blue] (1,1.5) circle (2.5pt);
			\path (1,-1) pic[red] {cross=2.5pt};
			\path (1,-1.5) pic[red] {cross=2.5pt};
			\path (1,-0.5) pic[red] {cross=2.5pt};
			\path (1,-2) pic[red] {cross=2.5pt};
			
			\draw [fill=blue] (1.5,-1) circle (2.5pt);
			\draw [fill=blue] (1.5,-1.5) circle (2.5pt);
			\draw [fill=blue] (1.5,-0.5) circle (2.5pt);
			\draw [fill=blue] (1.5,-2) circle (2.5pt);
			\path (1.5,0) pic[red] {cross=2.5pt};
			\path (1.5,0.5) pic[red] {cross=2.5pt};
			\path (1.5,1) pic[red] {cross=2.5pt};
			\path (1.5,1.5) pic[red] {cross=2.5pt};
			\end{tikzpicture}
		\end{center} 
	\end{minipage} 
	\hfill
	\begin{minipage}{0.25\linewidth} 
		\begin{center}
			\begin{tikzpicture}[scale = 1, line cap=round,line join=round,x=1.0cm,y=1.0cm]
			\path (-0.4,3.1) pic[red] {cross=2.5pt};
			\path (-0.74,2.84) pic[red] {cross=2.5pt};
			\path (-0.96,2.5) pic[red] {cross=2.5pt};
			\path (-0.66,2.36) pic[red] {cross=2.5pt};
			\path (-0.36,2.62) pic[red] {cross=2.5pt};
			\path (-0.34,2.24) pic[red] {cross=2.5pt};
			
			\draw [fill=blue] (-0.02,2.6) circle (1.5pt);
			\draw [fill=blue] (-0.06,3.12) circle (1.5pt);
			\draw [fill=blue] (0.,2.22) circle (1.5pt);
			\draw [fill=blue] (0.22,2.9) circle (1.5pt);
			\draw [fill=blue] (0.32,2.34) circle (1.5pt);
			\draw [fill=blue] (0.66,2.68) circle (1.5pt);
			\draw [fill=blue] (-0.46,4.) circle (1.5pt);
			\draw [fill=blue] (-0.96,4.02) circle (1.5pt);
			\draw [fill=blue] (-1.44,4.02) circle (1.5pt);
			\draw [fill=blue] (-1.46,3.68) circle (1.5pt);
			\draw [fill=blue] (-1.46,3.2) circle (1.5pt);
			\draw [fill=blue] (-1.44,2.8) circle (1.5pt);
			\draw [fill=blue] (-1.44,2.34) circle (1.5pt);
			\draw [fill=blue] (-1.42,1.72) circle (1.5pt);
			\draw [fill=blue] (-1.42,1.26) circle (1.5pt);
			\draw [fill=blue] (-0.96,1.22) circle (1.5pt);
			\draw [fill=blue] (-0.5,1.22) circle (1.5pt);
			\draw [fill=blue] (-0.56,1.68) circle (1.5pt);
			\draw [fill=blue] (-1.04,1.68) circle (1.5pt);
			\draw [fill=blue] (-1.04,2.14) circle (1.5pt);
			\draw [fill=blue] (-1.12,2.96) circle (1.5pt);
			\draw [fill=blue] (-0.78,3.34) circle (1.5pt);
			\draw [fill=blue] (-1.14,3.62) circle (1.5pt);
			\draw [fill=blue] (-0.44,3.6) circle (1.5pt);
			\draw [fill=blue] (-0.74,2.) circle (1.5pt);
			
			\path (0.44,3.64) pic[red] {cross=2.5pt};
			\path (-0.1,4.) pic[red] {cross=2.5pt};
			\path (-0.08,3.48) pic[red] {cross=2.5pt};
			\path (0.52,3.26) pic[red] {cross=2.5pt};
			\path (0.48,3.94) pic[red] {cross=2.5pt};
			\path (0.92,3.82) pic[red] {cross=2.5pt};
			\path (0.92,3.32) pic[red] {cross=2.5pt};
			\path (1.32,3.8) pic[red] {cross=2.5pt};
			\path (1.32,3.44) pic[red] {cross=2.5pt};
			\path (1.32,3.12) pic[red] {cross=2.5pt};
			\path (0.98,3.) pic[red] {cross=2.5pt};
			\path (1.12,2.7) pic[red] {cross=2.5pt};
			\path (0.76,2.3) pic[red] {cross=2.5pt};
			\path (0.38,1.9) pic[red] {cross=2.5pt};
			\path (-0.04,1.9) pic[red] {cross=2.5pt};
			\path (-0.04,1.34) pic[red] {cross=2.5pt};
			\path (0.26,1.38) pic[red] {cross=2.5pt};
			\path (0.86,1.38) pic[red] {cross=2.5pt};
			\path (0.68,1.94) pic[red] {cross=2.5pt};
			\path (1.16,2.) pic[red] {cross=2.5pt};
			\path (1.16,2.4) pic[red] {cross=2.5pt};
			\path (1.36,1.46) pic[red] {cross=2.5pt};
			\path (0.58,1.62) pic[red] {cross=2.5pt};
			\end{tikzpicture}
		\end{center} 
	\end{minipage} 
	
	Маша по жизни оптимистка. Поэтому она увидела не дополнительные траты на ремонт, а четыре задачи классификации. И теперь в её голове вопрос: сколько минимально нейронов нужно, чтобы эти задачи решить? 
\end{problem}


%%%-------------------------------------------
\begin{problem}{(универсальный регрессор)}
	Маша доказала Паше, что у неё всё в полном порядке с логикой. Теперь она собирается доказать ему, что с помощью однослойной нейронной сетки можно приблизить любую непрерывную функцию от одного аргумента $f(x)$ со сколь угодно большой точностью\footnote{\url{http://neuralnetworksanddeeplearning.com/chap4.html}}.  
	
	\textbf{Hint:}  Вспомните, что любую непрерывную функцию можно приблизить с помощью кусочно-линейной функции (ступеньки). Осознайте как с помощью пары нейронов можно описать такую ступеньку. Соедините все ступеньки в сумму с помощью выходного нейрона. 
\end{problem}


%%%%-------------------------------------------
\begin{problem}{(число параметров)}
	Та, кому принадлежит машин лёрнинг собирается обучить нейронную сеть для решения задачи регрессии, На вход в ней идёт $12$ переменных, в сетке есть $3$ скрытых слоя. В пером слое $300$ нейронов, во втором $200$, в третьем $100$. 
	
	\begin{itemize}
		\item[a)] Сколько параметров предстоит оценить Маше?  Сколько наблюдений вы бы на её месте использовали? 
		\item[b)] Что Маша должна сделать с внешним слоем, если она собирается решать задачу классификации на два класса и получать на выходе вероятность принадлежности к первому классу? 
		\item[c)]  Что делать Маше, если она хочет решать задачу классификации на $K$ классов? 
	\end{itemize}
\end{problem}

\end{document}